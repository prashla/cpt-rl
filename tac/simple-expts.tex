The main purpose of this numerical section is to justify two things: Firstly we believe that "CPT-functional" tends to react differently the way as expectation do to the change of parameters, specifically, we want to show that there are families of random variables, $X_\theta$ such that $\argmax_{\theta} \mathbb{E}\left( X_\theta \right)$ is quite distinct from 
$\argmax_{\theta} \mathbb{C} \left( X_\theta \right)$. Secondly we want to justify that our estimator been proposed in algorithm 
\ref{alg:holder-est} is asymptotically consistent.

The CPT-functional focused on this section is aligned with the form proposed in \eqref{eq:cpt-general}, with $u^+(x) = \max(x,0)$ and 
$u^-(x) = \min(x,0)$, $\lambda = 0.25$. 
More specifically,
\begin{align}
\C_{u,w}\left(X\right) &= \intinfinity w^+\left(\Prob{(X^+)^{0.88}>z}\right) dz\nonumber \\
&\quad - 0.25 \intinfinity w^-\left(\Prob{(X^-)^{0.88}>z}\right) dz \label{eq:CPT-numerical-section} , 
\end{align}
Where $w^+(p) = w^-(p) = \frac{p^{0.6}}{(p^{0.6} + (1-p)^{0.6})^{1/0.6}}.$ 



\subsection{Comparison between CPT and expectation}
Firstly we estimate the CPT-functional and expectation of a sequence of Gaussian r.v. $X_i$, distributed as $N(\mu_i, \sigma_i)$, where
$\mu_i$ is an increasing sequence of $[2.5, 2.6, \ldots 4.4, 4.5]$, and $\sigma_i$ is a decreasing sequence of $[5, 4.9, \ldots 1.1, 1]$.
Among all the 21 r.v.s, $X_{21}$ has the largest expectation $E(X_{21}) = 4.5$, but $X_1$ generates the largest CPT-value. See table
\ref{tab:gaussCPT}.


% \begin{table}
% \centering
% \begin{tabular}{|c|c|c|}
% \toprule
% $\mu$ & $\sigma$ & CPT-value $\C(X^{\mu,\sigma})$\\
% \midrule
% \textbf{2.5} & \textbf{6.0} & \textbf{1.8305}\\
% 2.6 & 5.8 & 1.8198\\
% 2.7 & 5.6 & 1.8008\\
% 2.8 & 5.4 & 1.7806\\
% 2.9 & 5.2 & 1.7677\\
% 3.0 & 5.0 & 1.7595\\
% 3.1 & 4.8 & 1.7397\\
% 3.2 & 4.6 & 1.7288\\
% 3.3 & 4.4 & 1.7158\\
% 3.4 & 4.2 & 1.7070\\
% 3.5 & 4.0 & 1.6922\\
% 3.6 & 3.8 & 1.6808\\
% 3.7 & 3.6 & 1.6739\\
% 3.8 & 3.4 & 1.6620\\
% 3.9 & 3.2 & 1.6560\\
% 4.0 & 3.0 & 1.6538\\
% 4.1 & 2.8 & 1.6529\\
% 4.2 & 2.6 & 1.6531\\
% 4.3 & 2.4 & 1.6601\\
% 4.4 & 2.2 & 1.6705\\
% \textbf{4.5} & \textbf{2.0} & \textbf{1.6848}\\
% \bottomrule
% \end{tabular}
% \caption{CPT-value for Gaussian distributed random variables with varying mean $\mu$ and variance $\sigma$. The r.v. (first row) with $(\mu,\sigma)=(2.6,6.0)$ has the highest CPT-value, while the r.v. on the last row has the highest expected value.}
% \label{tab:gaussCPT}
% \end{table}
% 
% \begin{table}
% \centering
% \begin{tabular}{|c|c|c|c|}
% \toprule
% $\xi$ & $\omega$ & CPT-value $\C(X^{\xi,\omega})$ & Expected value\\
% \midrule
% -1.00 & 5.0 & 1.7995 & 0.7841\\
% -0.90 & 4.8 & 1.7405 & 0.8128\\
% -0.80 & 4.6 & 1.7069 & 0.8414\\
% -0.70 & 4.4 & 1.6748 & 0.8700\\
% -0.60 & 4.2 & 1.6405 & 0.8987\\
% -0.50 & 4.0 & 1.5878 & 0.9273\\
% -0.40 & 3.8 & 1.5625 & 0.9559\\
% -0.30 & 3.6 & 1.5215 & 0.9846\\
% -0.20 & 3.4 & 1.4747 & 1.0132\\
% -0.10 & 3.2 & 1.4481 & 1.0418\\
% 0.00 & 3.0 & 1.4007 & 1.0705\\
% 0.10 & 2.8 & 1.3666 & 1.0991\\
% 0.20 & 2.6 & 1.3293 & 1.1277\\
% 0.30 & 2.4 & 1.3044 & 1.1564\\
% 0.40 & 2.2 & 1.2691 & 1.1850\\
% 0.50 & 2.0 & 1.2380 & 1.2136\\
% 0.60 & 1.8 & 1.2074 & 1.2423\\
% 0.70 & 1.6 & 1.1790 & 1.2709\\
% 0.80 & 1.4 & 1.1592 & 1.2996\\
% 0.90 & 1.2 & 1.1396 & 1.3282\\
% 1.00 & 1.0 & 1.1417 & 1.3568\\
% \bottomrule
% \end{tabular}
% \caption{CPT-value for skewed normal distributed random variables with shape $\alpha=0.5$ and varying location $\xi$ and scale $\omega$.} 
% \label{tab:skewNormalCPT}
% \end{table}

 \begin{figure}[h]
   \centering
   \begin{tabular}{c}
   \subfigure[Gaussian distributed r.v.s with parameters mean $\mu$ and variance $\sigma$] 
   {
   \scalebox{0.65}{\begin{tikzpicture}
   \begin{axis}[width=13cm,height=6.5cm,legend pos=south east,
          %  grid = major,         
            axis lines=middle,
           % grid style={dashed, gray!30},
            xmin=-1,     % start the diagram at this x-coordinate
            xmax=5,    % end   the diagram at this x-coordinate
            ymin=-1,     % start the diagram at this y-coordinate
            ymax=8,   % end   the diagram at this y-coordinate
           % axis background/.style={fill=white},
            ylabel={\large Variance $\sigma$},
            xlabel={\large Mean $\mu$},
            x label style={at={(axis cs:4.7,-2)}},
            y label style={at={(axis cs:-1.5,4.8)}},
%             xticklabels=\empty,
%             yticklabels=\empty
            ]
               
               \path[name path=diagplus] (axis cs:0.5,6) -- (axis cs:2.5,2);
               \draw (axis cs:0.5,6) -- (axis cs:2.5,2);
                \path[name path=xaxisplus] (axis cs:0.5,2) -- (axis cs:2.5,2);
               \draw (axis cs:0.5,2) -- (axis cs:2.5,2);
                 \path[name path=yaxisplus] (axis cs:0.5,2) -- (axis cs:0.5,6);
		\draw (axis cs:0.5,2) -- (axis cs:0.5,6);
               
 \addplot [green!40]  fill between[of= diagplus and xaxisplus];
 
%  \node at (axis cs:  1.2,3) {Feasible region};
         %%%%% CPT optimal point
\node at (axis cs:  2,7) {\makecell{\bf CPT-value \\ \bf optima}};
         \path[->] (axis cs:1.5,7.3) edge [out=180,in=90] (axis cs:0.5,6);

         %%%%% Expected optimal point
          \node at (axis cs:  4,2) {\makecell{\bf Expected value\\\bf optima}};
         \path[->] (axis cs:3.3,2.3) edge [out=180,in=0] (axis cs:2.5,2);
   \end{axis}
   \end{tikzpicture}}
   }
   \\
\subfigure[Skewed normal distributed r.v.s with fixed shape $\alpha=0.5$ and varying location $\xi$ and scale $\omega$] 
   {
   \scalebox{0.65}{\begin{tikzpicture}
   \begin{axis}[width=13cm,height=6.5cm,legend pos=south east,
          %  grid = major,         
            axis lines=middle,
           % grid style={dashed, gray!30},
            xmin=-3,     % start the diagram at this x-coordinate
            xmax=3,    % end   the diagram at this x-coordinate
            ymin=-1,     % start the diagram at this y-coordinate
            ymax=8,   % end   the diagram at this y-coordinate
           % axis background/.style={fill=white},
            ylabel={\large Location $\xi$},
            xlabel={\large Scale $\omega$},
            x label style={at={(axis cs:2.5,-2)}},
            y label style={at={(axis cs:-1.5,7)}},
%             xticklabels=\empty,
%             yticklabels=\empty
            ]
               
               \path[name path=diagplus] (axis cs:-1,5) -- (axis cs:1,1);
               \draw (axis cs:-1,5) -- (axis cs:1,1);
                \path[name path=xaxisplus] (axis cs:-1,1) -- (axis cs:1,1);
               \draw (axis cs:-1,1) -- (axis cs:1,1);
                 \path[name path=yaxisplus] (axis cs:-1,1) -- (axis cs:-1,5);
		\draw (axis cs:-1,1) -- (axis cs:-1,5);
               
 \addplot [green!40]  fill between[of= diagplus and xaxisplus];
 
         %%%%% CPT optimal point
\node at (axis cs:  -2,3) {\makecell{\bf CPT-value\\ \bf optima}};
         \path[->] (axis cs:-2,3.4) edge [out=180,in=90] (axis cs:-1,5);

         %%%%% Expected optimal point
          \node at (axis cs:  2,4) {\makecell{\bf Expected value\\\bf optima}};
         \path[->] (axis cs:1.8,4.4) edge [out=180,in=10] (axis cs:1,1);
   \end{axis}
   \end{tikzpicture}}
   } 
   \end{tabular}
\caption{Optimal expected and CPT-values within the feasible region (shaded green) for two different distributions.}
\label{fig:normal-cpt}
\end{figure}


Our second example is concerned with skew normal distribution, 
$sn(\xi, \omega, \alpha)$, where the mean equals 
$\xi + \omega \delta \sqrt{\frac{2}{\pi}}$, and variance equals 
$\omega^2(1 - \frac{2\delta^2}{\pi})$, with $\delta = \frac{\alpha}{1 + \alpha^2}$.
We decide to work on 21 r.v.s as well, denoted as $Y_i \sim sn(\xi_i, \omega_i, \alpha_i)$.
We will vary $\xi$ from $0$ to $2$ with ascending order and evenly gap, and vary $\omega$ from $5$ to $1$ with descending order with evenly gap. 
The table \ref{tab:skewNormalCPT} exemplifies that the optimal-points of CPT-functional and expectation are very distinct as well. 

\subsection{Justifying consistency of CPT estimator}
To numerically justify that our CPT-estimator is asymptotically consistent, we design the following steps:
Firstly we use gaussian-quadrature to numerically calculate the integral \eqref{eq:CPT-numerical-section} with underline r.v. $X$ be skew normal distributed as $sn(2,1,2)$. Notice that when the density of $X$ is known and $w^+(p)$, $w^-(p)$ are differentiable, the CPT-value \eqref{eq:CPT-numerical-section} could be formulated as
\begin{align}
\C_{u,w}\left( X \right) & = \intinfinity z^{0.88} (w^{+})^\prime \left(1 - F_{X^+}\left(z\right)\right) f_{X^+}\left(z\right)dz \\
&\quad - 0.25 \intinfinity (-z)^{0.88} (w^{-})^\prime \left(1 - F_{X^-}\left(z\right)\right) f_{X^-}\left(z\right) dz.
\end{align}
To show that CPT estimator is getting closer to the numerical integral, we propose total $100$ simulation phases indexed from $1$ to $100$. In each phase i , we generate 10 i.i.d. estimators $\C^j_{n_i}\left(X\right)$ of $\C_{u,w}\left( X\right)$($j$ from 1 to 10), according to the scheme of algorithm \ref{alg:holder-est} with $n_i$ samples generated from $X \sim sn(2,1,2)$. Additionally, $n_i$ ranges from $100$ to $1000,000$. Furthermore, in each phase $i$, we calculate the deviation of each estimator $\C^j_{n_i}\left(X\right)$ from numerically integrated value. We collect all the deviations and calculate their mean and margin of error , displayed in figure \ref{fig:cpt-est}. Margin of error stands for half length of t-confidence interval of those deviations.

%%%%%%%%%%%%%%%%%%%%%%%%%%%%%%%%%%5
%%%% Error bars 
%%%%%%%%%%%%%%%%%%%%%%%%%%%%%%%%%%%%
\newcommand{\errorband}[5][]{ % x column, y column, error column, optional argument for setting style of the area plot
\pgfplotstableread[col sep=tab, skip first n=2]{#2}\datatable
    % Lower bound (invisible plot)
    \addplot [draw=none, stack plots=y, forget plot] table [
        x={#3},
        y expr=\thisrow{#4}-2*\thisrow{#5}
    ] {\datatable};

    % Stack twice the error, draw as area plot
    \addplot [draw=none, fill=gray!40, stack plots=y, area legend, #1] table [
        x={#3},
        y expr=4*\thisrow{#5}
    ] {\datatable} \closedcycle;

    % Reset stack using invisible plot
    \addplot [forget plot, stack plots=y,draw=none] table [x={#3}, y expr=-(\thisrow{#4}+2*\thisrow{#5})] {\datatable};
}

\begin{figure}
    \centering
\scalebox{0.8}{\begin{tikzpicture}
      \begin{axis}[
	xlabel={Sample size},
	ylabel={Estimation error},
       %legend entries={
	 %,
        %estimator,
        %},
        %legend pos=north east,
				grid,grid style={gray!30}
      ]

      \errorband[red!50!white, opacity=0.3]{results/estimation_accuracy.txt}{0}{1}{4}
      \addplot [thick, red] table [x index=0, y index=1,col sep=tab] {results/estimation_accuracy.txt};
      \end{axis}
      \end{tikzpicture}}
			\caption{The deviations and its confidence intervals created in each phase }
      \label{fig:cpt-est} 
			\end{figure}
















