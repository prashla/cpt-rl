The region:
As of normal distribution $N(\mu, \sigma)$, we set up the feasible region to be a triangle composed with points
$(0.5,2), (0.5,6), (2.5,2)$. The expectation takes maximum apparently $N (2.5, 2)$. 
We try to find approximately the point where CPT-functional takes the maximum through out the triangle region,
We propose Grid search within the triangle region with each given point in the 
grid has $0.05$ distance from its adjacent points.

After numerically calculating all the CPT values at the grid points, we found out that  $N(0.5, 6)$ returns the maximum CPT-value with $2.6467$

In contrast, the r.v. $N (2.5, 2)$ only returns CPT value $2.3681$

Our second example is concerned with skew normal distribution, 
$X_{(\xi, \omega, \alpha)} \sim sn(\xi, \omega, \alpha)$, where the mean equals 
$\xi + \omega \delta \sqrt{\frac{2}{\pi}}$, and variance equals 
$\omega^2(1 - \frac{2\delta^2}{\pi})$, with $\delta = \frac{\alpha}{1 + \alpha^2}$.
As of skew normal distribution, we firstly fix the shape parameter $\alpha$ to be 0.5. And set up the feasible region of parameters $(\xi, \omega) $ to be the triangle composed with $(-1,1), (1,1), (-1,5)$.  Again we try to find the points returning the largest expectation and the largest CPT.
It turns out that the point $(-1,5)$ returns the largest CPT-value, and $\C(X_{(-1,5,0.5)}) = 2.3031$, in the meanwhile,
$\mathbb{E}(X_{-1,5,0.5}) = 0.7841$.
On the contrary, the point $(1,1)$ has the largest expectation $\mathbb{E}(X_{1,1,0.5}) = 1.3568$, but the CPT value of the same r.v. is only been approximately to be $1.2532$
