We consider a traffic signal control application where the aim is to optimize the delay experienced by road users. maximize the average happiness of the road users

\subsection{Simulation Setup}  
%We consider a SSP version of an example\footnote{A similar example has been considered in \cite{chow2014algorithms}.} for buying a house at the optimal price. Suppose the house is priced at $x_k$ at any instant $k$ and at the next instant, the price either goes down to $\left(x_k \times C_{down}\right)$ w.p. $p_{down}$ or goes up to $\left(x_k\times C_{up}\right)$ w.p. $1-p_{down}$. The actions are to either wait (denoted $w$), which results in a holding cost $h$ or to buy (denoted $b$) at the current price. The horizon is capped at $T$, with a terminal cost $x_T$.  The goal is to minimize the total cost defined as $ 
%D^{\theta}(x^0)= \sum_{k=0}^\tau \left(I_{\{a_k =b \} }x_k+I_{\{a_k =w \} } h\right) + I_{\{\tau=T\}} x_T$, where $\tau =  \{k | \theta(x_k)=1 \} \wedge T$.
%We set $T=20, h=0.1, C_{up}=2, C_{down}=0.5$, and $x_0=1$.  
%
 %\begin{figure*}
    %\centering
     %\begin{tabular}{cc}
%\begin{subfigure}[b]{0.5\textwidth}
%\tabl{c}{\scalebox{0.7}{\begin{tikzpicture}
%\begin{axis}[
%ybar={2pt},
%legend pos=north east,
%legend image code/.code={\path[fill=white,white] (-2mm,-2mm) rectangle
%(-3mm,2mm); \path[fill=white,white] (-2mm,-2mm) rectangle (2mm,-3mm); \draw
%(-2mm,-2mm) rectangle (2mm,2mm);},
%ylabel={\bf Expected value},
%xlabel={\bf Probability $\bm{p_{down}}$},
%xtick=data,
%ytick align=outside,
%xticklabel style={align=center},
%ymin=0.1,
%ymax=1.4,
%bar width=14pt,
%nodes near coords,
%grid,
%grid style={gray!30},
%width=11cm,
%height=9.5cm,
%]
%\addplot[fill=red!20]  table[x index=0, y index=1, col sep=comma] {../results/Valueiteration.txt} ;
%\addlegendentry{Value iteration}
%\end{axis}
%\end{tikzpicture}}\\[1ex]}
%\caption{Value iteration}
%\label{fig:vi}
%\end{subfigure}
%&
%\begin{subfigure}[b]{0.5\textwidth}
%\tabl{c}{\scalebox{0.7}{\begin{tikzpicture}
%\begin{axis}[
%ybar={2pt},
%legend pos=north east,
%legend image code/.code={\path[fill=white,white] (-2mm,-2mm) rectangle
%(-3mm,2mm); \path[fill=white,white] (-2mm,-2mm) rectangle (2mm,-3mm); \draw
%(-2mm,-2mm) rectangle (2mm,2mm);},
%ylabel={\bf CPT-Value},
%xlabel={\bf Probability $\bm{p_{down}}$},
%xtick=data,
%ytick align=outside,
%xticklabel style={align=center},
%bar width=14pt,
%ymin=0.1,
%ymax=1.4,
%nodes near coords,
%grid,
%grid style={gray!30},
%width=11cm,
%height=9.5cm,
%]
%\addplot[fill=blue!20]  table[x index=0, y index=1, col sep=comma] {../results/twospsa_cpt.txt} ;
%\addlegendentry{CPT-SPSA-N}
%\end{axis}
%\end{tikzpicture}}\\[1ex]}
%\caption{Second-order SPSA for CPT-value}
%\label{fig:cpt2spsa}
%\end{subfigure}
%\\
%\begin{subfigure}[b]{0.5\textwidth}
    %\tabl{c}{\scalebox{0.7}{\begin{tikzpicture}
%\begin{axis}[
%ybar={2pt},
%legend pos=north east,
%legend image code/.code={\path[fill=white,white] (-2mm,-2mm) rectangle
%(-3mm,2mm); \path[fill=white,white] (-2mm,-2mm) rectangle (2mm,-3mm); \draw
%(-2mm,-2mm) rectangle (2mm,2mm);},
%ylabel={\bf Expected value},
%xlabel={\bf Probability $\bm{p_{down}}$},
%xtick=data,
%ytick align=outside,
%xticklabel style={align=center},
%ymin=0.1,
%ymax=1.8,
%bar width=14pt,
%nodes near coords,
%grid,
%grid style={gray!30},
%width=11cm,
%height=9.5cm,
%]
%\addplot[fill=yellow!30]  table[x index=0, y index=1, col sep=comma] {../results/SPSADETERMINISTIC.txt} ;
%\addlegendentry{NoCPT-SPSA-G}
%\end{axis}
%\end{tikzpicture}}\\[1ex]}
%\caption{SPSA for regular value function}
%\label{fig:nocptspsa}
%\end{subfigure}
%&
%\begin{subfigure}[b]{0.5\textwidth}
%\tabl{c}{\scalebox{0.7}{\begin{tikzpicture}
%\begin{axis}[
%ybar={2pt},
%legend pos=north east,
%legend image code/.code={\path[fill=white,white] (-2mm,-2mm) rectangle
%(-3mm,2mm); \path[fill=white,white] (-2mm,-2mm) rectangle (2mm,-3mm); \draw
%(-2mm,-2mm) rectangle (2mm,2mm);},
%ylabel={\bf CPT-Value},
%xlabel={\bf Probability $\bm{p_{down}}$},
%xtick=data,
%ytick align=outside,
%xticklabel style={align=center},
%bar width=14pt,
%ymin=0.1,
%ymax=1.4,
%nodes near coords,
%grid,
%grid style={gray!30},
%width=11cm,
%height=9.5cm,
%]
%\addplot[fill=green!20]  table[x index=0, y index=1, col sep=comma] {../results/SPSACPT.txt} ;
%\addlegendentry{CPT-SPSA-G}
%\end{axis}
%\end{tikzpicture}}\\[1ex]}
%\caption{First-order SPSA for CPT-value}
%\label{fig:cptspsa}
%\end{subfigure}
%\end{tabular}
%\caption{Performance of policy gradient algorithms with/without CPT for different down probabilities of the SSP}
%\label{fig:perf}
%\end{figure*}
%
%
  %
%
%\paragraph{Implementation:} On this example, we implement the first-order CPT-SPSA-G and the second-order CPT-SPSA-N algorithms. For the sake of comparison, we also apply value iteration to the SSP example described above. 
%Note that value iteration requires knowledge of the model, while our CPT based algorithms estimate CPT-value using simulated episodes.
%We implement the algorithm from \cite{bhatnagar2004simultaneous} for the SSP example described in the numerical experiments of the main paper. The latter, henceforth referred to as NoCPT-SPSA-G, is an SPSA-based scheme that optimizes the traditional value function objective in a discounted MDP setting and we make a trivial adaptation of this algorithm for the SSP setting.
%For CPT-SPSA-G and NoCPT-SPSA-G, we set $\delta_n = 1.9/n^{0.101}$ and $\gamma_n = 1/n$, while for CPT-SPSA-N, we set $\delta_n=3.8/n^{0.166}$ and $\gamma_n=1/n^{0.6}$. For all algorithms, we set each entry of the initial policy $\theta_0$ to $0.1$. For CPT-value estimation, we simulate $1000$ SSP episodes, with the SSP horizon $T$ set to $20$. All algorithms are run with a budget of $1000$ samples, which implies $500$ iterations of CPT-SPSA-G and $333$ iterations of CPT-SPSA-N. The results presented are averages over $500$ independent simulations. For CPT-SPSA-G/CPT-SPSA-N, 
%the weight functions $w^+$ and $w^-$ are set to $p^{0.6}/(p^{0.6}+(1-p)^{0.6})$, while the utility functions are identity maps. 
%
%
%
%\subsection{Results} Figures \ref{fig:vi}--\ref{fig:nocptspsa} present the value function computed using value iteration and NoCPT-SPSA-G, while Figures \ref{fig:cptspsa}--\ref{fig:cpt2spsa} present the CPT-value $V^{\theta_{end}}(x^0)$ for CPT-SPSA-G and CPT-SPSA-N, respectively. The performance plots are for various values of $p_{down}$, the probability of house price going down. 
%From Figure \ref{fig:vi}, we notice that the variations in expected total cost is larger in comparison to that in CPT-value. Figure \ref{fig:nocptspsa} implies that a similar observation about variation of expected value holds true for NoCPT-SPSA-G algorithm from \cite{bhatnagar2004simultaneous}. While it is difficult to plot the entire policies, for the expected value minimizing algorithms it was observed that there were drastic changes in the policies with a change of $0.01$ in $p_{down}$, while PG/CPT-SPSA-N resulted in randomized policies that smoothly transitioned with changes in $p_{down}$.
%As motivated in the introduction, these plots verify that CPT-aware SPSA algorithms are less sensitive to the model changes as compared to the expected value minimizing algorithms. It is also evident that the second-order CPT-SPSA-N gives marginally better results than its first-order counterpart CPT-SPSA-G.
 %Finally, what is not shown is that the CPT-value obtained for PG/CPT-SPSA-N is much lower than that obtained for NoCPT-SPSA-G, thus making apparent the need for specialized algorithms that incorporate CPT-based criteria.
%
