For the sake of notational simplicity, we let $X$ denote the r.v. $X^\theta$, i.e., where the parameter $\theta$ is assumed to be fixed for the purpose of CPT-value estimation in this section. 

\paragraph{On integrability}
Observe that the first integral in \eqref{eq:cpt-mdp}, i.e., 
\begin{align}
\label{eq:1st-int-cpt}
\int_0^{+\infty} w^+(P(u^+(X)>z)) d z
\end{align}
may diverge even if the first moment of random variable $u^+(X)$ is finite. 
For example, suppose $U$ has the tail distribution function
$$P(U>z)  = \frac{1}{z^2}, z\in [1, +\infty),$$
and $w^+(z)$ takes the form $w(z) = z^{\frac{1}{3}}$. Then, the integral \eqref{eq:1st-int-cpt} with respect to the distorted tail, i.e.,
$$
\int_1^{+\infty} \frac{1}{z^{\frac{2}{3}}} dz
$$
does not even exist. A similar argument applies to the second integral in \eqref{eq:cpt-mdp} as well.

To overcome the above integrability issues, we make different assumptions on the weight and/or utility functions. In particular, we assume that the weight functions $w^+, w^-$ are either 
\begin{inparaenum}[\bfseries (i)]
\item Lipschitz continuous, or
\item \holder continuous, or
\item locally Lipschitz.
\end{inparaenum}
We devise a scheme for estimating \eqref{eq:cpt-mdp} given only samples from $X$ and show that, under each of the aforementioned assumptions, our estimator (presented next) converges almost surely. 
We also provide sample complexity bounds assuming that the utility functions are bounded.

% \subsubsection*{Main results}
% As mentioned earlier, to overcome integrability issues, we make the following assumption:\\[1ex]
% \textbf{Assumption (A1).}  The weight functions $w^+, w^-$ are Lipschitz with common constant $L$, and 
% $u^+(X)$ and $u^-(X)$ both have bounded first moments\\[1ex]
% % If the weight function is not Lipschitz, then the integrals using the distribution of $X$ in CPT-value may not even be finite. 
% We make the following assumptions on the utility functions:\\[1ex]
% \textbf{Assumption (A2).}  The utility functions $u^+(X)$ and $u^-(X)$ are continuous and increasing.\\[1ex]
% \todoj{Check if increasing is necessary}
% \textbf{Assumption (A2').}  In addition to (A2), the utility functions $u^+(X)$ and $u^-(X)$ are bounded above by $M<\infty$.\\[1ex]
% For the convergence rate results below, we require (A2'), while (A2) is sufficient to prove asymptotic convergence.
% 
% The following result shows that the estimate \eqref{eq:cpt-est} converges to the true CPT value almost surely and at the (nearly) canonical Monte Carlo asymptotic rate. 
% \begin{proposition} (\textbf{Asymptotic convergence and rate.})
% \label{thm:asymp-conv}
% Under (A1) and (A2), we have
% \begin{align}
% \widehat V_n(X) \rightarrow \C(X) \text{ a.s. as } n \rightarrow \infty.
% \end{align}
% If we assume (A2'), then we have
% $$
% \limsup_{n\rightarrow \infty} \sqrt{\frac{n}{2 \ln \ln n}} ||\hat{V_n}(X)-\C(X)||_{\infty} 
% \leq  LM \quad \text{a.s.}
% $$
% \end{proposition}
% \begin{proof}
%  See Section \ref{appendix:cpt-est}.
% \end{proof}
% 
% While the above result establishes that \eqref{eq:cpt-est} is an unbiased estimate in the asymptotic sense, it is important to know the rate at which the estimate in \eqref{eq:cpt-est} converges to the CPT-value. 
% The following sample complexity result shows that $O\left(\frac{1}{\epsilon^2}\right)$ number of samples are required to be $\epsilon$-close to the CPT-value in high probability.
% \begin{proposition}(\textbf{Sample Complexity})
% \label{thm:dkw}
% Under (A1) and (A2'), for any $\epsilon, \delta >0$, we have
% \begin{align*}
% P(|\widehat V_n(X) - \C(X)|\le\epsilon) \geq  1 - \delta, \,\,\, \forall n \geq \frac{2 L^2 M^2}{\epsilon^2} ln \frac{4}{\delta}.
% \end{align*}
% \end{proposition}
% \begin{proof}
%  See Section \ref{appendix:cpt-est}.
% \end{proof}

%%%%%%%%%%%%%%%%%%%%%%%%%%%%%%%%%%%%%%%%%%%%%%%%%%%%%%%%%%%%%%
%%%%%%%%%%%%%%%%%%%%%%%%%%%%%%%%%%%%%%%%%%%%%%%%%%%%%%%%%%%%%%
\subsection{Estimation scheme for \holder continuous weights}
Recall the H\"{o}lder continuity property first in definition 1:
\begin{definition}
{\textbf{\textit{(H\"{o}lder continuity)}}}
If $0 < \alpha \leq 1$, a function $f \in C([a,b])$ is said to satisfy
a H\"{o}lder condition of order $\alpha$ (or to be H\"{o}lder continuous
of order $\alpha$) if
\[
\sup_{x \neq y} \frac{| f(x) - f(y) |}{| x-y |^{\alpha}} \leq C .
\]
\end{definition}

In order to ensure integrability of the CPT-value \eqref{eq:cpt-mdp}, we make the following assumption:\\[1ex]
\textbf{Assumption (A1).}  
The weight functions $w^+, w^-$ are H\"{o}lder continuous with common order $\alpha$. Further,
$\exists \gamma \le \alpha \text{   s.t,  }$ 
$$\int_0^{+\infty} P^{\gamma} (u^+(X)>z) dz < +\infty \text{ and }\int_0^{+\infty} P^{\gamma} (u^-(X)>z) dz < +\infty$$

The above assumption ensures that the CPT-value as defined by \eqref{eq:cpt-mdp} is finite - see Proposition \ref{prop:Holder-cpt-finite} in Section \ref{sec:holder-proofs} for a formal proof.


\paragraph{Approximating CPT-value using quantiles:}
Let $\xi^+_{\frac{i}{n}}$ denote the $\frac{i}{n}$th quantile of the r.v. $u^+(X)$. Then, it can be seen that (see Proposition \ref{prop:holder-quantile} in Section \ref{sec:holder-proofs})
\begin{align}
\label{eq:holder-quant-motiv}
\lim_{n \rightarrow \infty} \sum_0^{n-1} \xi^+_{\frac{i}{n}} \left(w^+\left(\frac{n-i}{n}\right)- w^+\left(\frac{n-i-1}{n}\right) \right) = \int_0^{+\infty} w^+(P(u^+(X)>z)) dz.
\end{align}

The identical property holds for the pairs $u^-(X)$, $\xi^-_{\frac{i}{n}}, w^-$ , with $\xi^-_{\frac{i}{n}}$ denote the 
$\frac{i}{n}$th quantile of the r.v. $u^-(X)$.

However, we do not know the distribution of $u^+(X)$ or $u^-(X)$ and hence, we develop a procedure that uses order statistics for estimating quantiles, which in turn assists in estimating the CPT-value along the lines of \eqref{eq:holder-quant-motiv}. The estimation scheme is presented in Algorithm \ref{alg:holder-est}.

\todoj[inline]{Need some text here to connect the quantile business to EDFs, i.e., to say that step 5 in Algo 1 is the same as estimating EDFs and then doing empirical integration using EDFs to estiamete CPT-value}

\begin{algorithm}
\caption{CPT-value estimation for \holder continuous weights}
\label{alg:holder-est}
\begin{algorithmic}[1]
\State Simulate $n$ random samples with distribution $X$, sort them and denoted the ordered sample as 
$X_{[1]}, X_{[2]}, \ldots X_{[n]}$.
\State Calculate $u^+(X_{[1]}),\ldots u^+(X_{[n]})$
\State Order the simulated samples and label them as follows: 
$u^+(X_{[1]}),\ldots,u^+(X_{[n]})$.
\State Use $u^+(X_{[i]}), i\in \mathbb{N}\cap (0,n)$ as an approximation for the $\frac{i}{n} th$ quantile of $u^+(X)$, i.e, $\xi_{\frac{i}{n}}, i\in \mathbb{N}\cap (0,n)$.
\State Denote the statistic 
$\overline \C_n^+:=\sum_{i=1}^{n-1} u^+(X_{[i]}) (w^+(\frac{n-i}{n})- w^+(\frac{n-i-1}{n}) )$
\State Repeat the procedure on the sequence $X_{[1]}, X_{[2]}, \ldots X_{[n]}$, with respect to the function $u^-$, 
and denote the statistic $\overline \C_n^-:=\sum_{i=1}^{n-1} u^-(X_{[i]}) (w^-(\frac{n-i}{n})- w^-(\frac{n-i-1}{n}) ) $
\State Return the statistic $\overline \C_n =\overline \C_n^+ - \overline \C_n^-$.
\end{algorithmic}
\end{algorithm}

\subsubsection*{Main results}
We make the following assumptions on the utility functions:\\[1ex]
\textbf{Assumption (A2).}  The utility functions $u^+(X)$ and $u^-(X)$ are continuous and increasing.\\[1ex]
\todoj{Check if increasing is necessary}
\textbf{Assumption (A2').}  In addition to (A2), the utility functions $u^+(X)$ and $u^-(X)$ are bounded above by $M<\infty$.\\[1ex]
For the convergence rate results below, we require (A2'), while (A2) is sufficient to prove asymptotic convergence.

\begin{proposition}(\textbf{Asymptotic convergence.})
\label{prop:holder-asymptotic}
Assume (A1) and also that $F^+(\cdot)$,$F^-(\cdot)$ - the distribution functions of $u^+(X)$, and $u^-(X)$ are Lipschitz continuous with constants $L^+$ and $L^-$, respectively on the interval $(0,+\infty)$, and 
$(-\infty, 0)$ . Then, we have that
\begin{align}
\lim_{n\rightarrow +\infty} 
\overline \C_n
=
\C(X)
 \text{   a.s.,}
\end{align}
where $\overline \C_n$ is as defined in Algorithm \ref{alg:holder-est} and $\C(X)$ as in \eqref{eq:cpt-general}.
\end{proposition}
\begin{proof}
See Section \ref{sec:holder-proofs}.
\end{proof}

While the above result establishes that $\overline \C_n$ is an unbiased estimate in the asymptotic sense, it is important to know the rate at which the estimate $\overline \C_n$ converges to the CPT-value $\C(X)$. 
The following sample complexity result shows that $O\left(\frac{1}{\epsilon^{2/\alpha}}\right)$ number of samples are required to be $\epsilon$-close to the CPT-value in high probability.

% 
% While the above proposition gives an asymptotic guarantee, in the following we provide a sample complexity result for Algorithm \ref{alg:holder-est}.
% For this result, we assume, as in the case of Lipschitz continuous weights, that the utility functions are bounded in addition to (A1').

\begin{proposition}(\textbf{Sample complexity.})
\label{prop:holder-dkw}
Assume (A1) and (A2'). Then, $\forall \epsilon, \delta$, we have
$$
P(\left |\overline \C_n- \C(X) \right| \leq  \epsilon ) > \delta\text{     ,} \forall n \geq \ln(\frac{1}{\delta})\cdot 
\frac{4L^2 M^2}{\epsilon^{2/\alpha}}.$$
\end{proposition}
\begin{proof}
Notice the the following equivalence:
$$\sum_{i=1}^{n-1} u^+(X_{[i]}) (w^+(\frac{n-i}{n}) - w^+(\frac{n-i-1}{n})) =  \int_0^M w^+(1-\hat{F^+_n}(x)) dx, $$
and also,
$$\sum_{i=1}^{n-1} u^-(X_{[i]}) (w^-(\frac{n-i}{n}) - w^-(\frac{n-i-1}{n})) =  \int_0^M w^-(1-\hat{F^-_n}(x)) dx, $$

where $\hat{F^+_n}(x)$ and $\hat{F^-_n}(x)$ is the empirical distribution of $u^+(X)$
and $u^-(X)$, defined as follows:
\begin{align}
{\hat F_n}^+(x)=&\frac{1}{n} \sum_{i=1}^n 1_{(u^+(X_i) \leq x)}, 
{\hat F_n}^-(x)=\frac{1}{n} \sum_{i=1}^n 1_{(u^-(X_i) \leq x)}.
\label{eq:edf}
\end{align}

The main claim follows from the equivalence mentioned above together with the well-known DKW inequality.
The detailed proof is available in Section \ref{sec:holder-proofs}.
\end{proof}

\subsubsection*{Special case: Lipschitz continuous weights}
\textbf{Assumption (A1').}  The weight functions $w^+, w^-$ are Lipschitz with common constant $L$, and 
$u^+(X)$ and $u^-(X)$ both have bounded first moments.\\[1ex]

\begin{corollary}[Lipschitz case]
Assume (A1') and (A2). Then, we have that 
$$\lim_{n\rightarrow +\infty} \overline \C_n = \C(X) \text{   a.s.}$$

In addition, if we assume (A2'), we have 
$$
P(\left |\overline \C_n- \C(X) \right| \leq  \epsilon ) > \delta\text{     ,} \forall n \geq \ln(\frac{1}{\delta})\cdot 
\frac{4L^2 M^2}{\epsilon^{2}}.
$$
\end{corollary}
\begin{proof}
 Setting $\alpha=\gamma=1$ in the proof of Proposition \ref{prop:Holder-cpt-finite}, it is easy to see that the CPT-value \eqref{eq:cpt-mdp} is finite. Thus, the claims regarding asymptotic convergence and sample complexity as special cases of Proposition \ref{prop:holder-asymptotic}--\ref{prop:holder-dkw}, with $\alpha=1$. 
\end{proof}



%%%%%%%%%%%%%%%%%%%%%%%%%%%%%%%%%%%%%%%%%%%%%%%%%%%%%%%%%%%%%%
%%%%%%%%%%%%%%%%%%%%%%%%%%%%%%%%%%%%%%%%%%%%%%%%%%%%%%%%%%%%%%
%%%%%%%%%%%%%%%%%%%%%%%%%%%%%%%%%%%%%%%%%%%%%%%%%%%%%%%%%%%%%%
\subsection{Estimation scheme for locally Lipschitz weights and discrete $X$}
\todop[inline]{Would be better if the background been put to the introduction part}
Here we assume that the r.v. $X$ is discrete valued. 
Let $p_i, i=1,\ldots,K,$ denote the probability of incurring a gain/loss $x_i, i=1,\ldots,K$, where 
$x_1\le \ldots \le x_l \le 0 \le x_{l+1} \le \ldots \le x_K$ and  let
\begin{align}
\label{eq:Fk}
 F_k = 
   \sum_{i=1}^k p_k  \text{ if   } k \leq l \text{ and }
   \sum_{i=k}^K p_k  \text{ if  }  k > l.
\end{align}
Then, the CPT-value is defined as 
\begin{small}
\begin{align*}
&\C(X) \!=\! (u^-(x_1)) w^-(p_1) 
\!+\!\sum_{i=2}^l u^-(x_i) \Big(w^-(F_i) - w^-(F_{i-1})\Big) \\
& + \sum_{i=l+1}^{K-1} u^+(x_i) \Big(w^+(F_i) - w^+(F_{i+1}) \Big)
 + u^+(x_K) w^+(p_K),
\end{align*} 
\end{small}
where $u^+, u^-$ are utility functions and $w^+, w^-$ are weight functions corresponding to gains and losses, respectively. The utility functions $u^+$ and $u^-$ are non-decreasing, while the weight functions are continuous, non-decreasing and have the range $[0,1]$ with $w^+(0)=w^-(0)=0$ and $w^+(1)=w^-(1)=1$. 

\subsubsection*{Estimation scheme} 
Let $X_1,\ldots,X_n$ be $n$ samples from the distribution of $X$. 
Define $\hat p_k:= \frac{1}{n} \sum_{i=1}^n I_{\{X_i =x_k\}}$ and 
\begin{align}
\label{eq:Fkhat}
 \hat F_k = 
   \sum_{i=1}^k \hat p_k  \text{ if   } k \leq l \text{ and }
   \sum_{i=k}^K \hat p_k  \text{ if  }  k > l.
\end{align}
Then, we estimate $\C(X)$ as follows:
\begin{small}
\begin{align*}
&\overline \C_n \!=\! 
u^-(x_1) w^-(\hat p_1) \!+\!\sum_{i=2}^l u^-(x_i) \Big(w^-(\hat F_i) - w^-( \hat F_{i-1})\Big) 
\nonumber\\
&
+ \sum_{i=l+1}^{K-1} u^+(x_i) \Big(w^+(\hat F_i) - w^+(\hat F_{i+1}) \Big) + u^+(x_K) w^+(\hat p_K). 
%\label{eq:cpt-discrete-est}
\end{align*}
\end{small}
%Because $\hat{p_k}$ converge a.e to $p_k=P(X_i=x_k)$, with $X_i$ be the ith sample of $X$, the above estimator is  strong consistent property by the continuous mapping theorem. 
%The following proposition presents a sample complexity result for the discrete-valued $X$ under the following assumption:\\

\noindent\textbf{Assumption (A1'').}  The weight functions $w^+(X)$ and $w^-(X)$ are locally Lipschitz continuous, i.e., for any $x$, there exist  $L_x< \infty$ and $\rho>0$, such that
$$| w^+(x) - w^+(y) | \leq L_x |x-y|, \text{ for all } y \in (x-\rho,x+\rho). $$

% \subsubsection*{Main result}
\begin{proposition}
\label{prop:sample-complexity-discrete}
Assume (A1''). Let $L=\max\{L_k, k=2...K\}$, where $L_k$ is the local Lipschitz constant of function $w^-(x)$ at points
$F_k$, where $k=1,\ldots,l$, and of function $w^+(x)$ at points $k=l+1,\ldots,K$. 
Let $M=\max\{u^{-}(x_k), k=1,\ldots,l\} \bigcup \{u^{+}(x_k), k=l+1,\ldots,K\}$ and $\rho =\min\{\rho_k\}$, where $\rho_k$ is half the length of the interval centered at point $F_k$ where (A3) holds with constant $L_k$.
Then, $\forall \epsilon>0,\delta >0$, we have 
\begin{align*}
\Prob{\left|
\overline \C_n -\C(X)
\right| \leq \epsilon} \! > \! 1\!-\!\delta, \forall n \ge \frac{1}{\kappa}\ln\!\left(\frac{1}{\delta}\right) \ln\left(\frac{4K}{M}\right)\!, 
\end{align*}
where $\kappa=\min(\rho^2, \epsilon^2/(KLM)^2)$.
\end{proposition}
% \begin{corollary}
% Under conditions of Proposition \ref{prop:sample-complexity-discrete}, we have
% $$
% \E \left|\overline \C_n- \C(X) \right|  \le  \frac{(2H M)^\alpha \Gamma\left(\alpha/2\right)}{n^{\alpha/2}}.$$
% \end{corollary}

In comparison to Propositions \ref{prop:holder-dkw} and \ref{prop:lipschitz}, 
observe that the sample complexity for discrete $X$ scales with the local Lipschitz constant $L$ and this can be much smaller than the global Lipschitz constant of the weight functions, or the weight functions may not be Lipschitz globally.  

\begin{proof}
 See Section \ref{sec:proofs-discrete}.
\end{proof}

A variant of Corollary \ref{cor:holder-dkw} can be claimed by integrating the high-probability bound in Proposition \ref{prop:sample-complexity-discrete} and we omit the details here.

%The detailed proofs of Propositions \ref{prop:holder-asymptotic}--\ref{prop:sample-complexity-discrete} are available in \cite{Pracheng2015cpt}.

